\documentclass[a4paper,10pt, oneside]{article}
\usepackage[utf8]{inputenc}
%\usepackage[french]{babel}
\usepackage{fullpage,graphicx,}
\usepackage{hyperref}
\usepackage[procnames]{listingsutf8}
\usepackage[svgnames]{xcolor}
\usepackage{fancyheadings}
\usepackage{pifont}
\usepackage{pgf}
\usepackage{tikz}
\usetikzlibrary{arrows,automata}
\pagestyle{fancy}

%%%%%%%%%%%%%%%%%%%%%%%%%%%%%%%%%%%%%%% 
\lstnewenvironment{algo}[1][]
{   
  \lstset{ %this is the stype
    frame=tB,
    numbers=left, 
    numberstyle=\tiny,
    basicstyle=\ttfamily\small,
    identifierstyle=\color{black}\bfseries,
    keywordstyle=\color{LightSeaGreen}\bfseries,
    keywords={ALGORITHME, PROGRAMME, ALGO, VAR, DONNEES, VARIABLES,
      DEBUT, FIN, SI, ALORS, FSI, SINON, ET,OU, NON, TQ, FTQ, FAIRE,
      POUR, FPOUR, ALLOUER, RENVOYER},
    numbers=left,
    stepnumber=1,
    xleftmargin=.04\textwidth
  }
}
{}

\lstset{
  language=C,
  basicstyle=\ttfamily\small,
  keywordstyle=\color{LightSeaGreen}\bfseries,
  commentstyle=\color{SeaGreen}\slshape,
  stringstyle=\color{IndianRed}\ttfamily,
  showstringspaces=false,	
  numbers=left, numberstyle=\tiny, stepnumber=0, numbersep=5pt, firstnumber=1,
  showspaces=false, showtabs=false, 
  backgroundcolor=\color{Lavender},
  procnamestyle=\color{DodgerBlue}
}


%%%
\newcounter{exo}
\setcounter{exo}{0}
\newenvironment{exercice}%
{\stepcounter{exo}%
\par\noindent%
\fcolorbox{white}{lightgray}{\textsc{Exercice \theexo.}}
}%
{\vspace{4mm}}
%%%
\fancyhf{}
%\rhead{Université de Toulon -L1 Maths/SI/PC-}
\cfoot{Compilation}
\lfoot{Nicolas Méloni}
\rfoot{\thepage}

\begin{document}
\title{I53 - Projet de fin de semestre\\Conception d'un compilateur ALGO / RAM}

\date{Licence 3 - 2022/2023}
\maketitle

\section*{Introduction}

Le but de ce TP est de construire un compilateur du langage
algorithmique vers la machine RAM du cours d'algorithmique de 2ème
année (\texttt{arc} pour
\textbf{A}lgo-\textbf{R}am-\textbf{C}ompiler). Pour cela nous
utiliserons conjointement \texttt{Flex} et \texttt{Bison} pour
construire l'arbre syntaxique abstrait du programme, une table de
symbole sous forme de double liste chaînée et un module chargé de
gérer l'arbre abstrait et de produire le code cible.

L'archive \texttt{ProjetARC.tar} contient une version élémentaire du
programme. Celui-ci peut être compilé à l'aide de la commande
\texttt{make}. La structure du projet est présentée ci-dessous. Une
étude du \texttt{makefile} permettra de bien comprendre la structure
générale du projet. Quelques exemples de programmes ALGO que le
compilateur doit être en mesure de gérer sont proposés dans le
répertoire \texttt{test}.


\begin{lstlisting}[]
>> ls *
makefile

bin:

doc:
arc_doc.tex

include:
asa.h  codegen.h  parser.h  ram_os.h  semantic.h  ts.h

lib:
RAM_OS.ram

obj:

src:
asa.c  codegen.c  lexer.lex  parser.y  semantic.c  ts.c

test:
asa  exemple1.algo  exemple2.algo  exemple3.algo  exemple4.algo  ts

\end{lstlisting}


La compilation du programme peut être décomposé en les étapes
suivantes:

\begin{enumerate}
\item \textbf{construction de l'arbre abstrait:} on utilise Bison et Flex pour
  construire un analyseur syntaxique et les fonctions du fichiers
  \texttt{asa.h} pour générer un arbre abstrait à partir des feuilles.
\item \textbf{Analyse sémantique:} on parcours l'arbre abstrait en
  profondeur pour remplir la table de symbole, vérifier les règles
  sémantiques (déclaration avant utilisaion, valeurs min/max des
  entiers etc) et calculer au cours des retours arrières le nombre
  d'instructions correspondant à chaque noeud.
\item \textbf{Génération de code:} on parcours l'arbre abstrait en
  profondeur et on utilise les informations de la table de symbole et
  les nombres d'instructions pour produire le code RAM.

\end{enumerate}

\subsection*{Table de symboles}

Les fichiers \texttt{ts.[ch]} sont fournis complets: ce sont les seuls
qui n'ont pas besoin d'être modifiés. La table de symbole est une
structure de double liste chaînée. La liste principale est une liste
de contextes représentant la portée d'un ensemble
d'identificateurs. Chaque sous liste est la liste des identificateurs
du contexte en question.

\begin{lstlisting}[language=C]
typedef struct symbole{
  char id[ID_SIZE_MAX];
  int type; 
  int size;
  int adr;
  struct symbole *next;
} symbole;

typedef struct contexte{
  char name[ID_SIZE_MAX];
  symbole * liste_symbole;
  struct contexte * next;
} contexte;
\end{lstlisting}

\subsection*{Arbre de syntaxe abstrait}

Afin de produire du code pour la machine RAM il est nécessaire de
construire une représentation abstraite du programme analysé. Cette
construction se fera à l'aide de la structure \texttt{struct asa}
définie dans le fichier \texttt{asa.h}. Un noeud de l'arbre sera
composé d'un type (correspondant chacun à une construction de la
grammaire) et d'une structure correspondant au type en question.

Le type des noeuds est simplement un type énuméré :

\begin{lstlisting}[language=C]
  typedef enum {typeNB, typeOP} typeNoeud;
\end{lstlisting}

et chaque type correspondra à une structure propre:

\begin{lstlisting}[language=C]
typedef struct {
  int val;
} feuilleNB;

typedef struct {
  int ope;
  struct asa * noeud[2];
} noeudOP;

typedef struct asa{
  typeNoeud type;
  int memadr;
  int codelen;
  union {
    feuilleNb nb;
    noeudOp op;
  };
} asa;
\end{lstlisting}

\section*{Norme du langage ALGO}

La norme d'un veritable langage est un document bien trop pour être
conçu avec rigueur dans un simple sujet de travaux pratiques. Nous
nous contenterons ici de décrire les prinpales caractéristiques du
langage.


\subsection*{Structure d'un programme}

Un programme ALGO est constitué d'un seul fichier source l'extension
\texttt{.algo}. Celui-ci doit comporter la définition de zéro, un ou
plusieurs alogrithmes, zéro une ou plusieurs variables ainsi qu'une
fonction principale nomée \texttt{PROGRAMME()} qui sera exécutée par
la machine cible. Celle-ci  renvoie une valeur entière en fin
d'exécution (0 par défaut).

\subsection*{Types de données}

\begin{enumerate}
\item \texttt{ENTIER}: nombre sur 16 bits (compris entre
  \texttt{ENTIER\_MIN}$=-2^{15}$ et \texttt{ENTIER\_MAX}$=2^{15}-1$. 
\item \texttt{TABLEAU}: suite de cases consécutives en mémoires,
  adressé à l'aide de l'opérateur d'incidage [ ]. Le premier élément
  est indicé par 0.
\item \texttt{POINTEUR}: Adresse d'une donnée. Obtenu par l'opérateur
  de déreférencement @.
  
\end{enumerate}

\subsection*{Portée des identifiants}

Chaque identificateur possède une portée définie au moment de sa
déclaration.

\begin{enumerate}
\item Un identificateur déclaré à l'intérieur d'une fonction n'est
  visible qu'à l'intérieur de celle-ci. 
\item Un identificateur déclaré en dehors de toute fonction est
  considéré de porté globale et est visible depuis n'importe quelle
  fonction.
\item Si un identificateur de portée globale porte le même nom qu'un
  identificateur au sein d'une fonction, l'identificateur global est
  ignoré.
\end{enumerate}


\section*{Langage ALGO}

\subsection*{Lexique du langage}


\begin{enumerate}
\item \textbf{Mots Clés}: \texttt{PROGRAMME, ALGO, DEBUT, FIN, VAR,
    TQ, FAIRE, FTQ, SI, ALORS, SINON, FSI, ET, OU, NON, VRAI,
    FAUX, LIRE, ECRIRE, ALLOUER, RENVOYER}
\item \textbf{Ponctuateurs}: \texttt{+ - * /  \% ( ) <-  = != <= >= , [ ] @ \textbackslash n} 
\item \textbf{Blancs}: \verb|' ' \t|
\item \textbf{Nombres}: entiers décimaux sans 0 superflus
\item \textbf{Identificateurs}: chaînes de caractères alpha-numériques
  commençcant par un caractère
\end{enumerate}

\subsection*{Grammaire du langage}


\begin{enumerate}
\item Un programme est suite d'algorithmes définis par le mot clé
  \texttt{ALGO} un identificateur est un liste de paramètres de type
  \texttt{ENTIER} (aucun marquage) ou \texttt{POINTEUR} (\texttt{id}
  précédé d'un \texttt{@}).
\item Le corps d'une fonction commence par un suite de zero, une ou
  plusieurs déclarations de variables (avec ou sans affectation) suivi
  du mot clé \texttt{DEBUT}, une série d'instructions et le mot clé
  \texttt{FIN} (on utilisera le ponctuateurs \texttt{\textbackslash n}
  comme séparateur d'instructions).
\end{enumerate}

La grammaire est celle manipulée en cours depuis la 1ère
année. L'algorithme suivant permet de voir a peu près toutes les
constructions nécessaires.

\begin{algo}
//echange les elements des cases d indices i et j
ALGO ECHANGER( @T,i,j)
VAR temp
DEBUT
	temp <- T[i]
	T[i] <- T[j]
	T[j] <- temp
FIN

// Renvoie l indice de la plus petite valeur du tableau T entre i et n	
ALGO Selection( @T, n, i )
VAR imin <- i
DEBUT
	i <- i+1
	TQ i < n FAIRE
	   SI T[imin] > T[i] ALORS
	      imin <- i
	   FSI
	   i <- i+1
	FTQ
	RENVOYER imin
FIN

PROGRAMME()
VAR taille, @T
VAR i, imin
DEBUT
	//stockage des donnees dans un tableau dynamique
	taille <- LIRE()
	ALLOUER( T, taille )
	i <- 0
	TQ i < taille FAIRE
	   T[i] <- LIRE()
	   i <- i+1
	FTQ

	//Tri selection
	i <- 0
	TQ i < taille FAIRE
	   ECHANGER( T, i, Selection(T, taille, i))
	   i <- i+1
	FTQ

	//Affichage du tableau trie
	i <- 0
	TQ i < taille FAIRE
	   ECRIRE(T[i])
	   i <- i+1
	FTQ
	
	RENVOYER 0
FIN
\end{algo}



\end{document}
